\chapter{Увод}\label{ch:\thechapter}

Када се спомену матурски радови, често су прва асоцијација теоријска излагања школског градива. Моја идеја је да овај рад има употребну вредност и ван контекста школе и понуди податке који су иначе тешко доступни, а имају својеврстан јавни значај. Подаци са уписа у средње школе су добар пример привидне транспарентности података -- комплетан опус је јавно доступан, али како би се из њих извукле иоле корисне информације потребно је уложити значајан труд у њихово прикупљање и обраду, тако да једино што преостаје је ослањање на званичне извештаје који због самог опсега података никако не могу пружити комплетан увид у ову тему која је од посебног значаја у тренутку када су реформе образовног система константно у жижи јавности.

Намера рада је да читав процес прикупљања и примера обраде спусти на \enquote{школски} ниво, тако да за његово схватање буде довољно разумевање градива које се предаје у средњим школама на смеровима за обдарене за математику. Услед обимности теме и величине самог пројекта, од теоријских основа ће бити покривено само оно што се непосредно користи у раду. Такође, фокус овог рада није на специфичним програмским језицима, већ на концепту и прецизно описаном поступку којим се до података долази, а комплетан изворни код је доступан у Додатку В.

Код је тестиран на Ubuntu и elementaryOS системима, али с обзиром на начелну портабилност свих коришћених алата, требало би истоветно да ради на свим дистрибуцијама Linux-а и на Windows и macOS оперативним системима. Конкретне команде су или идентичне или потпуно аналогне за све платформе.

\section{Отворени подаци}

Према дефиницији са Портала отворених података Републике Србије, отворени подаци су они који подаци који су \emph{слободно доступни, приступачни, машински читљиви и доступни у отвореним форматима} \citep{opendatadef}. Овакви подаци су посебно значајни када је реч о подацима који могу бити од интереса јавности. Уместо ослањања на постојеће извештаје, отворени подаци допуштају верификацију резултата, њихове допуне и исправке од стране било кога.

\subsection{Упис у средње школе у Србији}

Иако Министартство просвете, науке и технолошког развоја Републике Србије има посебан сајт посвећен отвореним подацима (\url{http://opendata.mpn.gov.rs}), њиме није обухваћен и систем за упис у средње школе. Наиме, за потребе уписа користи се посебна платформа која се налази на \url{http://upis.mpn.gov.rs}. Овако објављени подаци се тешко могу назвати отвореним: иако су слободно доступни и приступачни, машинскo читање је могуће али је нетривијално и они нису доступни у отвореним форматима. 

Платформа је првенствено намењена будућим средњошколцима и мануелном приступу. На њој се може видети све што се бодује при упису у средњу школу (оцене из последња три разреда, успех на завршном испиту, успех на пријемном испиту, награде на такмичењима за оне године када се то рачунало, бодови добијени на рачун афирмативних мера), као и основни детаљи за сваку од основних и средњих школа чији је оснивач Република Србија. Свим подацима појединачно се може приступити преко сајта, али не постоји начин за преузимање свих података заједно, нити могућност да се они на било који начин машински анализирају. Такође, сајт се сваке године ажурира новим подацима, а од старих јавно доступни остају само поједине просечне вредности из претходне године. Ови подаци се не могу упоредити са ранијим циклусима уписа, а ни сами просеци нису довољни за било какве озбиљније анализе.

Завод за вредновање квалитета образовања и васпитања издаје годишњи Извештај о реализацији и резултатима завршних испита на крају основног образовања и васпитања (доступни на \texttt{\href{https://ceo.edu.rs/\%D0\%B8\%D0\%B7\%D0\%B2\%D0\%B5\%D1\%88\%D1\%82\%D0\%B0\%D1\%98\%D0\%B8-\%D0\%BE-\%D1\%80\%D0\%B5\%D0\%B0\%D0\%BB\%D0\%B8\%D0\%B7\%D0\%B0\%D1\%86\%D0\%B8\%D1\%98\%D0\%B8-\%D0\%B8-\%D1\%80\%D0\%B5\%D0\%B7\%D1\%83\%D0\%BB\%D1\%82\%D0\%B0\%D1\%82\%D0\%B8\%D0\%BC/}{https://ceo.edu.rs/извештаји-о-реализацији-и-резултатим}}) \citep{izvestaj15}\citep{izvestaj16}\citep{izvestaj17}, који иако је добра почетна тачка, фокусира се на завршни испит и популацију углавном дели по основној школи и, чешће, територијално, уместо нпр. по уписаној школи или по подручју рада уписаног смера (оваква методологија је разумљива с обзиром да је циљ Извештаја анализа завршног испита, али иде у прилог становишту да један извештај годишње, ма колико опширан, никако не може бити довољан). У одељку \ref{ch:obradasql} је показано да управо подела по подручју рада има смисла, али нигде (па ни у овом раду) овај феномен није детаљније истражен.

\section{Циљ и структура рада}

Циљ овог рада је да покаже једну од метода за прикупљање јавно доступних информација, као и на који начин се прикупљени подаци могу употребити. Такође, крајњи резултат овог рада је база података која се може употребити за даља истраживања. Први део рада се тиче прикупљања података и његово чување у фајловима и у релационој бази података. Како је сајт са подацима које се односе на упис доживео велики редизајн и модернизацију у току 2017. године, постоје два начина за преузимање података које овај програм подржава у зависности од варијанте сајта са којег се подаци преузимају. 

Други део рада се односи на обраду података и описује начине за манипулисање са подацима на два начина: учитавањем фајлова са подацима и користећи изграђену базу података, као и алгоритам за симулацију уписа. Сви прикупљени подаци се односе на ученике који су уписали средњу школу сем ако није другачије назначено, односно циљ рада није анализирање комплетне популације основаца у датом периоду, пошто је упитно да ли је такав подухват тренутно уопште могућ, а ако јесте онда захтева далеко већи ангажман у односу на то колико користи може да донесе. Конкретaн број ученика чији подаци недостају овде дат је у одељку \ref{subs:osn_obrada}.