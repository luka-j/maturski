\chapter{Дискусија}\label{ch:\thechapter}

\section{Будући рад -- идеје за испитивање}

У претходне две главе су показани само неки основни примери. Наравно, користећи исте податке, могу се написати много комплексније анализе које би испитивале разне феномене. Стандардизовањем резултата завршног испита они се могу уједначити и бити упоредиви и кроз различите генерације. На основу овако спојене три популације, може се хипотетисати који све спољашњи фактори и колико утичу на резултате (нпр. број становника у датом месту, просечни социоекономски статус, број ученика школе, штрајк просветних радника, итд.). Овај принцип је применљив како и за основне школе, тако и за уписане смерове.

Могу се испитивати повезаности резултата на завршном испиту са оценама у школи и у којој мери они одговарају једни другим. Нешто од овога је рађено у склопу Извештаја о завршном испиту \citep{izvestaj15}\citep{izvestaj16}\citep{izvestaj17}, али детаљније анализе комбинованог теста изостају (парадоксална је ситуација да Завод који саставља ове испите у извештајима тврди да комбиновани тест нема репрезентативни примерак питања услед малог обима и стога се не може поуздано анализирати, а да се у исто време повећава број бодова који он носи). Такође, не постоје никакве анализе које ученике групишу по некој особини уписаног смера, иако смо у одељку \ref{ch:obradasql} видели да постоје немале разлике и у овом аспекту.

Симулације могу помоћи при одговору на хипотетичка питања. Оквир је довољно робустан да сем основног скалирања (нпр. \emph{како би се променили резултати ако би завршни испит носио више или мање бодова}) дозволи и различито бодовање у зависности од смера -- рецимо, могуће је симулирати један критеријум уписа за гимназије (нпр. већи акценат на комбинованом тесту и генералним оценама) а други за електротехничке школе (нпр. већи акценат на математици и физици). Анализом тако хипотетички уписаних ученика може се закључити каква се селекција врши и да ли она позитивно или негативно утиче на остатак система. Овим се претпоставља да систем бодовања не утиче на резултате и листу жеља, што иако нереално, у неким случајевима не мора нужно да има превелик утицај.

\section{Закључак}
Сваки од показаних примера може бити одељак за себе. Овај рад не пружа одговоре на суштинска питања као што су \emph{зашто} су уочени такви подаци или \emph{како} су они настали, већ само константује да одређени феномени постоје. Међутим, овде су по први пут ови подаци изложени на систематичан и машински читљив начин. Значај овог пројекта је што омогућава да се на та питања да одговор и ван званичних извештаја, и надам се да ће у будућности бити користан било коме ко пожели да се бави анализом на овај начин прикупљених података. Настојао сам да овде приказани програми буду што је могуће више одвојени и употребљиви независно једни од других, као и да је база података систем за себе који се може користити без познавања скрепера који преузима податке и сервера који их трансформише и учитава у базу. Овиме је омогућено да се база укључује у друге пројекте, независно од платформе и намене.

Такође, ово је леп пример да није практично ограничити се на само један језик или платформу за било који нетривијалан пројекат. Изложена су три програма која се надовезују један на други и чине једну велику целину, и кроз рад су уведена три значајно различита програмска језика. Сваки од њих има своје предности и мане, и у зависности од потребе треба проценити који приступ ће најбрже дати најпрецизније резултате. Ово уопштено важи за програмирање као дисциплину: као што добар мајстор неће чекићем шрафити шраф, тако и добар програмер треба да зна да процени који је најбољи алат за дати посао.

Најважније, циљ овог рада је био приказати колико информација нам је заправо на располагању. Интернет је неисцрпно добро, тако да чак и када администрација закаже при правилном отварању података, док год они постоје у било којoj форми они нису \enquote{заробљени}. Концептуално једноставни програми могу бити спона између \emph{de jure} јавних информација и \emph{de facto} отворених података и тиме отворити многа наизглед запечаћена врата.