\chapter{Закључак}\label{ch:\thechapter}

Сваки од показаних примера може бити одељак за себе. Овај рад не пружа одговоре на суштинска питања као што су \emph{зашто} су уочени такви подаци или \emph{како} су они настали, већ само константује да одређени феномени постоје. Међутим, значај овог пројекта је што омогућава да се на та питања да одговор и ван званичних извештаја, и надам се да ће у будућности бити користан било коме ко пожели да се бави анализом на овај начин прикупљених података.

Такође, ово је леп пример да није практично ограничити се на само један језик или платформу за било који нетривијалан пројекат. Изложена су три програма која се надовезују један на други и чине једну велику целину, и кроз рад су уведена три значајно различита програмска језика. Сваки од њих има своје предности и мане, и у зависности од потребе треба проценити који приступ ће најбрже дати најпрецизније резултате. Ово уопштено важи за програмирање као дисциплину: као што добар мајстор неће чекићем шрафити шраф, тако и добар програмер треба да зна да процени који је најбољи алат за дати посао.

Најважније, циљ овог рада је био приказати колико информација нам је заправо на располагању. Интернет је неисцрпно добро, тако да чак и када администрација закаже при правилном отварању података, док год они постоје у било кој форми они нису \enquote{заробљени}. Концептуално једноставни програми могу бити спона између \emph{de jure} јавних информација и \emph{de facto} отворених података и тиме отворити многа наизглед запечаћена врата.