\chapter{Појам скреповања}\label{ch:\thechapter}
\vspace*{-8mm}
\textit{Овде се, као и у остатку рада, због стила и недостатка боље формулације реч ученик употребљава у граматичком мушком роду. Овај израз, наравно, обухвата и све ученице.} %Politička (rodna, kakogod) korektnost over 9000. todo Formulisati ovo lepše
\vspace{5mm}

\section{Шта је скреповање}

Скреповање Веба (енгл. web scrapping, у дословном преводу Веб стругање) подразумева било који поступак којим се подаци из формата који је читљив људима пребацују у формат читљив машинама \citep{boeing2017new}. Веб странице су најчешће писане у HTML-у и CSS-у (од енгл. Hypertext Markup Language и Cascaded Style Sheets), уз опционе процедуралне елементе кроз Javascript, које претраживач чита и приказује кориснику. Садржај се по правилу налази у оквиру HTML или као што ћемо видети у неким случајевима унутар Javascript кода, док CSS носи информацију о изгледу странице. Овај процес је оптималан када су циљна група људи, међутим ови језици су далеко експресивнији и носе значајан вишак информација за потребе анализирања самих података (нпр. начин на који је све визуелно организовано на страници), који треба одстранити. 

Најочигледнији метод скреповања је ручни -- тако што човек копира, налепљује и организује одговарајуће податке тако да се они касније могу лако учитати користећи неки програм. Ово је, међутим, ретко практично, и користи се само када не постоји други избор, или за мање корекције самих података. Други најједноставнији метод је тражење образаца у тексту/коду и извлачење фрагмената са одговарајућих места (видети, на пример, grep алат на Unix-базираним системима и појам регуларних израза). И овај поступак је незадовољавајући када је реч о страницама које се највећим делом састоје од HTML кода и као такве имају велики број сличних образаца у тексту које је тешко разликовати. Значајно бољи начин је парсирање HTML-а на начин сличан оном како то претраживачи раде.

\subsection{HTML и DOM}

HTML је језик чији су основни елементи тагови који одређују тип података и могу носити опционалне атрибуте као што су јединствена идентификациона вредност (id) или класа (class), а унутар којих се налази садржај (конкретна синтакса нам у овом случају није значајна). Када се тај код машински \enquote{прочита} и интерпретира, добијамо Објектни модел документа (енгл. Document Object Model, или DOM), који у меморији хијерархијски представља све елементе у облику стабла са кореним чвором \emph{document}. Ова структура је много лакша за манипулисање \citep{gupta2003dom}, и овде ће се највише користити за долажење до садржаја конкретног елемента на страници. Објектни модел документа је апстрактни интерфејс који није везан ни за један конкретан језик или платформу \citep{w3domdef}, али ће се у контексту овог рада увек односити на резултат парсирања HTML странице.

\subsection{Легалност скреповања}

Када је реч о прикупљању података без експлицитне дозволе, често се помиње легални аспект и да ли је и када то законом дозвољено. Најчешћи аргумент против овакве технике је угрожавање ауторских права или кршење Услова коришћења сајта, и на то треба посебно обратити пажњу. 

У Србији, колико је мени познато, није било већих случајева из ове области, а закон који уређује област је управо Закон о ауторским и сродним правима \citep{autorskaprava}. Конкретно, у случају овог рада, прикупљани подаци нису оригинална духовна творевина аутора (чл. 2 Закона), него подаци настали у раду органа јавне власти како је дефинисано у чл. 2 Закона о слободном приступу информацијама од јавног значаја \citep{zospi} у које се по чл. 3 истог закона убрајају и све државне основне и средње школе у Србији, што их чини ван опсега ауторског права а сврстава их у јавно доступне информације. Такође, нигде на сајту не постоје експлицитно наведени услови употребе и ограничења која би се односила на скреповање, укључујући и robots.txt фајл који се конвенционално користи за истицање правила понашања свих посетилаца који нису људи, већ скрипте и програми у било каквом облику (видети Стандард за искључење робота \citep{robotstxt}).

\section{Извор података}

Као извор података за скреповање користи се \url{http://upis.mpn.gov.rs}, који је званичан портал Министарства просвете, науке и технолошког развоја посвећен упису у средње школе чији је оснивач Држава, аутономна покрајина или јединица локалне самоуправе \citep{upismpn}. На порталу се могу пронаћи подаци о сваком уписаном ученику (у анонимизираној форми, тј. под шифром) и свакој основној и средњој школи која је део система, али насупрот тврђењу на почетној страници које обећава \emph{све статистичке податке који могу помоћи у одабиру школе} \citep{upismpn} сматрам да се на њему може видети само делић статистичих података који се коришћењем свих изложених података може извести. Овај рад ће пружити примере за неке од њих и начинити базу доступну свима који имају могућности и вољу да дају свој допринос.

\section{Структура портала посвећеног упису у средње школе}

Портал је доживео редизајн 2017. године, стога је било потребно направити скрепер (енгл. scraper, у значењу \enquote{алат за стругање}) у две варијанте, односно ажурирати постојећи тако да може да \enquote{разуме} и нову верзију. Основна структура портала је остала иста. Одељак \textit{Подаци о ученицима} омогућава претрагу конкретног ученика по шифри и ово нам за аутоматско преузимање података није од користи пошто може постојати и до милион шифара за мање од седамдесет хиљада ученика колико упише средњу школу сваке године (ову тврдњу ћемо доказати у другом делу, у Глави \ref{ch:5}). Одељак \textit{Основне школе} пружа основне информације о свакој основној школи као што су име, место, број ученика и просечни бројеви бодова на крају разреда и на завршном испиту; ови подаци се прикупљају парцијално од 2016. године, а у пуном облику од 2017-те. 

\textit{Средње школе и образовни профили} је најкориснији одељак за скрепер. Он обухвата, за сваки образовни профил, тј. смер средње школе, списак шифара ученика, број бодова и круг уписа за сваког уписаног ученика. Када знамо шифру ученика, можемо је користити да приступимо страници која садржи податке о конкретном ученику, као што су оцене из свих предмета у завршне три године које се бодују при упису и бодови на завршном испиту. И када се порталу приступа ручно, текст којим је исписана шифра је хиперлинк до странице ученика, тако да је без много муке могуће одратити верификацију појединих записа.