\chapter{Десктоп клијент}\label{ch:A}

Пошто је оригинално сервер био постављен јавно на интернету, имало је смисла писати десктоп клијент који би се повезивао на њега. Клијент је stand-alone Java апликација која омогућава графичко састављање упита и исцртавање scatter plot-ова. Иако у склопу сервера постоји слична функционалност, главна предност овог приступа је једноставност употребе и point-and-click интерфејс који дозвољава само смислене упите (са опционим напредним модом који дозвољава и текстуално састављање упита), тако да је приступачан свима. Код за плотовање је део апликације, и писан са циљем да може брзо и ефикасно да додаје велики број тачака (реда величине $10^5$) произвољних пречника на панел и подржава hover и транспарентност боја.

Сервер у тренутку писања није онлајн, тако да сама апликација није функционална. Међутим, пошто су у питању пројекти отвореног кода, сваки део се може прилагодити и изменити. Конкретно, очекује се да се URL сервера налази у \code{rs.luka\allowbreak.upisstats\allowbreak.desktop\allowbreak.io\allowbreak.Network\allowbreak\#URL\_BASE}. Овај пројекат је додатни пример како се нов код може интегрисати у мали екосистем представљен у овом раду.

Изворни код је јавно доступан на мом GitHub-у: \url{https://github.com/luq-0/UpisDesktop}. Пројекат користи Gradle као build system, који је доступан на \url{https://gradle.org/} у тренутку писања захтева Javа верзију 8 или новију.